\documentclass[a4paper]{article}\usepackage[dvips]{graphicx}

\begin{document}


\centerline{{\huge NeXus for Synchrotrons }}
\vspace{1cm}

\centerline{NeXus Workshop, May 10-12, 2010, Paul Scherrer Institute }
\vspace{1.5cm}



21 IT personnel and scientists from various synchrotron and neutron source around the 
world gathered at PSI to attend a workshop on the use of the NeXus data format for 
synchrotron instrumentation. The purpose of the workshop thus can be summarised:


\begin{itemize}\item Disseminate about information about the NeXus data format and how it works
\item Discuss the use  NeXus for synchrotron instrumentation
\item Create a wishlist for the NeXus International Advisory Committee (NIAC) with 
  changes needed to NeXus in to make it usable for synchrotron instrumentation.
\end{itemize}The organisers thank all participants for many open and constructive 
discussions. 
\vspace{1cm}


\leftline{{\huge {\bf Workshop Outcomes }}}
\vspace{.5cm}
There was a general agreement that NeXus has much to offer to the synchrotron 
community. But it is important to note that synchrotron people mean, when 
speaking of NeXus, HDF following NeXus conventions. There is no interest in the other 
NeXus supported file formats.  Beyond an efficient physical file format, HDF-5, NeXus offers a 
standardized way to store and document data. In addition NeXus provides 
an API and utilities which simplify the use of NeXus files. However, in order 
to become useful for synchrotron applications some changes to NeXus are required.



\section{Multi Method Instruments }

Synchrotron beamlines often utilise several different detectors and detector types in order 
to combine multiple techniques in simultaneous measurements. NeXus currently asks for separate 
NXentry groups to be written for each technique. This is good if one measurement is written to 
a file. However, there is a second requirement that multiple scans, multiple measurements, possibly 
a whole log of an experimental session is written to one NeXus file. Then having different 
techniques in different NXentries will make the files difficult to understand as the relationship
between different measurments is lost.  Thus, in order to keep the data from these multiple 
techniques together, it is desirable to have the ability to write it all into a single NXentry in 
a NeXus. The current NeXus application definitions refer to the same names and paths and so 
there are many name collisions when trying to satisfy two application definitions in one NXentry in 
a file. The ability to combine application definitions could be enabled by modifying the application 
definitions to refer to new and separate groups inside the main NXentry of the NeXus file that refer 
to the particular application/technique name and which contains all of the data (or links to it) that 
is relevant to that application/technique. For an example experiment that involves a combination of 
SAS and Fluorescence, the proposed NeXus structure could look like:


\begin{verbatim}

entry:NXentry/
  definition = "NXSas, NXFluo"
  user:NXuser/
  sample:NXsamle/
  instrument:NXinstument/
    SASdet:NXdetector/
    fancyname:NXdetector/
    fancyname2:NXdetector/
    ...
  SAS:NXsubentry/
    definition = "NXSas"
    instrument:NXinstrument/
      detector:link to SASdet
    data:NXdata/
  Fluo:NXsubentry/
    definition = "NXFluo"
    instrument:NXinstrument/
      detector:link to fancyname
      detector2:link to fancyname2
    data:NXdata/
\end{verbatim}

In the above NeXus tree, the entire beamline state could be stored in entry/instrument and then any subset 
of this that is relevant to the SAS or Fluorescence techniques would then be linked within the 
entry/SAS/instrument and the entry/Fluo/instrument groups as defined by the current application definitions 
with a minor change in the heirarchy. The advantages of this approach are:


\begin{itemize}\item Only minor changes from current practice.
\item The only name collisions to worry about are the names of the applications/techniques themselves.
\item Application definitions need not be concerned with the names and paths that other application 
  definitions proscribe.
\item The paths for each application remains well defined and an analysis program for either technique can 
  find the relevant data without having to understand the other techniques present in the file. Further, 
  the same analysis programs can read the multi-technique files in the same way (i.e. with the same code) 
  exactly the same as they read single-technique files.
\item A user inspecting the data manually can find all the relevant information for a particular analysis 
  in the one group and so doesn't need to understand the entire beamline.
\end{itemize}
One drawback of this approach is that the beamline staff would have to define many links when configuring 
the data acquisition software. However, this is necessary work regardless of how the data is saved since 
the user must be informed of how the different instrument components and detectors relate to the various 
analyses anyway. In fact, NeXus and the above proposal simplifies this task by clearly documenting in a 
formal manner where the relevant information can be read.


Some examples of beamlines that would benefit from this proposal include:


\begin{itemize}\item Fluorescence + Absorption + Diffraction (Beamline L, Doris, Beamline P06, PETRA III).
\item PX + fluorescence: In PX often a fluorescence-signal is recorded, especially for 
  SAD-/MAD-measurements (PETRA beamlines P11 P13 P14).
\item SAXS+fluorescence: fluorescence is often used as a second signal in SAXS (PETRA P03).
\item SAXS + ellipsometry: Beamline BW4, DORIS
  
\end{itemize}Summarizing this discussion, the suggestion is to allow NXentry or possibly new NXsubentry groups 
underneath NXentry. Each of which can adhere to a different application definition. All participants 
agreed that a good means of handling multi technique instruments in NeXus is essential for the 
adoption of NeXus at synchrotron facilities. This is a MUST HAVE!



\section{Scaled Data }

The data rates possible at synchrotron facilities and the new pixel detectors test current computing 
 technology to their limits. There may not be enough time to scale or convert data on the fly. Thus the 
acceptance of the Scaled Data proposal from the CIF study is another MUST HAVE. Furthermore it was 
suggested to rename the linearity attribute to transform, not to be confused with the transform field of 
the CIF coordinate suggestions,  and to add additional transformations, namely for a polynomial 
data scaling and a conversion from anything to energy using the Bragg equation. 





\section{Simplified Hierarchy }

In order to satisfy the requirements of the beamline scientist an additional, simplified 
NeXus hierarchy was proposed:


\begin{verbatim}

entry:NXentry
   measurement:NXmeasurement
      positions:NXpositioners
      scalars:NXscalar
      images:NXimagedata
\end{verbatim}
  
Please note that this is an example how a NXmeasurement group may look like. The general feeling was 
to allow much freedom in NXmeasurement and standardize later on if a common pattern emerges. 
The meaning is that the NXpositioners groups contains a list of all constants and motor positions, 
NXscalar arrays of all parameters varied during the scan and NXimageData what has been captured 
during a scan or measurement. This structure is for the expert, the instrument scientist, who knows 
his instrument by heart and wishes to be able to plot anything against anything in his instrument. 
NXmeasurement is not meant to stand alone but is to be augmented with further NXsubentries 
containing the data in proper NeXus notation and hierarchy. It was pointed out that such a 
simplified view can be generated by a data analysis application from a full NeXus NXinstrument view. 
It remains unclear if this is a must have or an optional feature. 



\section{Special Techniques }

The tomography people among thge participants got together and discussed the NeXus tomography 
application definitions. This discussion will continue beyond the workshop. An intermediate 
result is that the sample positions needs more detail then in the current application definition in 
order to allow for more exact reconstructions. 


There was also a review of the SAS definition by some participants. The results were:


\begin{itemize}\item All the SAS techniques should be joined in one application definitions
\item The above require more fields to describe both the sample and detector 
  position in more detail
\item The position of the beam stop needs to be added
\item A few more minor additions.
\end{itemize}The NXsas definition will soon be updated to accomodate this feedback.



A couple of neutron people among the participants and discussed how to store S(Q,om) 
processed data. Agreement was reached to store such data as arrays of Qh, Qk, Ql, En, I. 
Though data may have some cohesion, in general after conversion the data is on a irregular 
grid and the data analysis or visualisation application will have to rebin the data anyway 
in order to generate a representation. This consensus will soon be cast into an application 
definition.
 



\section{Additional Suggestions and Notes  }

There are some wishes about documentation: For binary packages it should be stated more 
clearly for which platform they are and what to do when they do not work. For example 
NeXus provides a Windows package which works on XP. But people with Vista or Windows 7 
may have problems. It should also be clearer how application definitions work with 
multiple detectors.


There is a wish to store the energy profile in NXmonochromator, NXsource and NXbeam. Suggested 
name: energy\_profile.  


Some more fields in NXdetector: x\_correction and y-correction. This is for storing the infromation 
where a pixel truly is. Technique for saying with which technique the detector is used. Live\_time 
is total\_counting\_time - dead\_time). 


There is a wish to annotate data with an attribute dimensionality which can have values of:
scalars, histogram, image, vertex. This to say that this data is a series of images, 
histograms etc. This helps a generic visualisation program to interpret the data properly. The 
string to give is part of an enumeration. 


NeXus should be more specific about the encoding for content strings, like in titles, 
sample names etc. A good suggestion is to specify UTF-8.   


The topic of validation was also stressed. The NeXus group is on the right track with the 
implementation of nxvalidate. 

A large number of participants enjoyed using the python interface and would appreciate if a more 
advanced python interface, like the one presented by Ray Osborn, would make it into the standard 
NeXus distribution.


The NeXus group was encouraged to seek inclusion of NeXus into Linux package repositories 
such as Debian or Fedora for easier installation. This is of course a man power problem.  





\end{document}

