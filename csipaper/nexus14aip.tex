% ****** Start of file aipsamp.tex ******
%
%   This file is part of the AIP files in the AIP distribution for REVTeX 4.
%   Version 4.1 of REVTeX, October 2009
%
%   Copyright (c) 2009 American Institute of Physics.
%
%   See the AIP README file for restrictions and more information.
%
% TeX'ing this file requires that you have AMS-LaTeX 2.0 installed
% as well as the rest of the prerequisites for REVTeX 4.1
%
% It also requires runnin. The commands are as follows:
%
%  1)  latex  aipsamp
%  2)  bibtex aipsamp
%  3)  latex  aipsamp
%  4)  latex  aipsamp
%
% Use this file as a source of example code for your aip document.
% Use the file aiptemplate.tex as a template for your document.
\documentclass[%
 aip,
%jmp,%
%bmf,%
rsi,
 amsmath,amssymb,
%preprint,%
 reprint,%
%author-year,%
%author-numerical,%
]{revtex4-1}

\usepackage{graphicx}% Include figure files
\usepackage{dcolumn}% Align table columns on decimal point
\usepackage{bm}% bold math
\usepackage{url}
%\usepackage[mathlines]{lineno}% Enable numbering of text and display math
%\linenumbers\relax % Commence numbering lines

\begin{document}

\preprint{AIP/123-QED}

\title{The NeXus Data Format}


\author{Mark K\"onnecke}
\affiliation{ 
Laboratory for Development and Methods\\Paul Scherrer Institute\\5232 Villigen-PSI\\Switzerland
}
\author{Frederick Akeroyd}
\affiliation{ISIS, Rutherford Appleton Labortaory, UK}

\author{Herbert J Bernstein}
\affiliation{imgCIF, Dowling College USA}

\author{Aaron Brewster}
\affiliation{Lawrence Berkeley Laboratory, USA}

\author{Bjoern Clausen}
\affiliation{Los Alamos National Laboratory, USA}

\author{Stephen Cottrell}
\affiliation{SICS, Rutherford Appleton Laboratory, UK}

\author{Jens Uwe Hoffmann}
\affiliation{Helmholtz Zentrum Berlin, Germany}

\author{Pete Jemian}
\affiliation{Advanced Photon Source, USA}

\author{David M\"annicke}
\affiliation{ANSTO Australia}

\author{Raymond Osborn}
\affiliation{Argonne National Laboratory, USA}

\author{Peter F. Peterson}
\affiliation{Spallation Neutron Source, USA}

\author{Tobias Richter}
\affiliation{Diamond Light Source, UK}

\author{Jiro Suzuki}
\affiliation{KEK, Japan}


\author{Bejmanin Watts}
\affiliation{Swiss Light Source, Paul Scherrer Institute, 5232 Villigen-PSI, Switzerland}

\author{Eugen Wintersberger}
\affiliation{DESY, Germany}

\author{Joachim Wuttke}
\affiliation{FRMII, JCNS, Germany}


\collaboration{NeXus International Advisory Committee}

\date{\today}% It is always \today, today,
             %  but any date may be explicitly specified

\begin{abstract}
NeXus is an effort by an international group of scientists to define 
 a common data exchange format for neutron, muon and x-ray experiments.   
NeXus is built on top of the scientific data format HDF-5 and adds domain 
specific rules for organizing data within HDF-5 files as well as a dictionary of well 
defined domain-specific field names. The NeXus data format has two purposes.  First, NeXus defines a
format that can serve as a container for all relevant data associated
with a beamline, an increasingly important function.  Second, NeXus
defines standards in the form of \emph{application definitions} for the
representation of information defining various experimental
techniques NeXus provided structures for raw experimental data as well as for processed data.  
\end{abstract}

%%\pacs{Valid PACS appear here}% PACS, the Physics and Astronomy
                             % Classification Scheme.
\keywords{NeXus, data format, HDF-5, x-ray, neutron,data analysis, data management}
\maketitle


\section{Introduction}
Increasingly, major neutron and x-ray facilities choose to store their data in the NeXus data format. 
Since 2006, NeXus\cite{nxold} has undergone substantial refocusing, 
refinement and enhancement as described in this paper.  

Historically, neutron and x-ray facilities choose to store their data in a plethora of 
home-grown data formats. This scheme has a number of drawbacks addressed by NeXus: 
\begin{itemize}
\item It makes the life of the traveling scientist more difficult then it needs to be as they have to deal with multiple files 
 in different formats, file converters and such in order to extract scientific information from the data.
 \item An unnecessary burden is imposed on data analysis software producers as they have to accommodate so many different formats.  
\item The whole idea of open access to data is sabotaged if the data is in a format which cannot be easily understood.
\item Modern high speed detectors produce data at such a rate that many older single image storage schemes become impractical and 
 an efficient container format is a necessity. 
\end{itemize}

The first necessity for a data format is a physical file format: how is the data written to disk? Rather than inventing  
yet another format, NeXus chose HDF-5\cite{hdf5} as the binary container format. HDF-5 is efficient, self describing, 
platform independent, in the public domain and well supported by both commercial and free software tools. 

NeXus adds to HDF-5:
\begin{itemize}
\item Rules for organizing domain-specific data within a HDF-5 file
\item A dictionary of documented domain-specific field names
\item Definitions of standards that can be validated
\end{itemize}

The development of NeXus is overseen by a committee, the NeXus International Advisory Committee (NIAC).

\section{NeXus Design Principles}


NeXus encourages its users to store all relevant data associated with their experiment or data analysis 
in a NeXus container file. This includes the state of the beamline, detector data, metadata, sample environment 
logs  and more.  The general idea is for users to have to read only one logical file. 
That enables the user to analyse data in ways yet unforeseen because the necessary additional 
fields have been stored. The burden of writing complete files is usually with the producing 
facility or software,  not the user. Thus this is no problem for the user. 

Consistent with the container file objective, it is always possible to add data, even non-NeXus data, 
to a NeXus data file without breaking NeXus. 

Of course, NeXus strives for platform independence, a self describing efficient file format, public 
domain specifications and tools. These aims were the reasons for choosing HDF-5 as the physical file format.

A NeXus container file may contain hundreds of fields. Often, all that is needed for data reduction or analysis are a small fraction 
of these - perhaps fewer than 20 fields. These necessary fields are dependent on the experimental technique and NeXus provides a way to 
describe a set of necessary fields as a NeXus application definition. Files can be validated against such an application definition
by checking for the presence of the required fields. This is the way in which NeXus expresses file standards. 

\section{NeXus File Hierarchies}
NeXus data files are organized into a hierarchy of groups which, in turn, can contain further groups or fields, 
very much like an internal file system. For an overview of the NeXus data file structure for raw experimental data see Table \ref{rawfile}.
\begin{table}
\caption{Overview of the structure of a NeXus raw data file \label{rawfile}}
\begin{tabular}{|p{1cm}|p{6.5cm}|}
\hline
\multicolumn{2}{|l|}{NXroot}\\
\hline
The root level of a NeXus file & 
\begin{tabular}{p{1.5cm}|p{4.8cm}}
\multicolumn{2}{l}{NXentry}\\
\hline
All data belonging to one scan or run. A given NeXus file
can contain multiple related scans or runs &
\begin{tabular}{p{1.5cm}|p{3.1cm}}
\multicolumn{2}{l}{NXinstrument}\\
\hline
The data needed to describe an instrument. Contains groups for
each relevant instrument component&
\begin{tabular}{p{3.0cm}}
NXsource\\
\hline
NXcollimator\\
\hline
NXattenuator\\
\hline
NXdetector\\
\hline
\ldots\\
\end{tabular}\\
\hline
\multicolumn{2}{l}{NXsample}\\
\multicolumn{2}{p{3.0cm}}{All the information about the sample}\\
\hline
\multicolumn{2}{l}{NXmonitor}\\
\multicolumn{2}{l}{Intensity monitor}\\
\hline
\multicolumn{2}{l}{NXuser}\\
\multicolumn{2}{l}{User information}\\
\hline
\multicolumn{2}{l}{NXdata}\\
\multicolumn{2}{p{3.0cm}}{Links to plottable data in the NXdetector group --
 one instance for each detector bank. This provides support for generating a
typical plot automatically}\\
\end{tabular}\\
\end{tabular}\\
\hline
\end{tabular}
\end{table}

In the following sections, we will describe some of the rules that define the overall structure of NeXus files. Many fine details 
of the NeXus format have been thoroughly discussed and are now well defined, but for the sake of brevity, we will not present an 
exhaustive view of NeXus here. A full listing of NeXus rules are given in the NeXus manual\cite{nxman}. Some examples of these 
additional rules include those governing:
\begin{itemize}
\item How axes are associated with data
\item That units must be given with the data
\item How data is to be stored in NeXus fields
\item How to describe array data which is not in ANSI C storage order
\end{itemize}



\subsection{NeXus Raw Data File Hierarchy}
The major focus of NeXus has been the recording of ``raw'' experimental data, i.e. information taken directly from the experimental 
equipment, or processed only as required to provide physically meaningful values.
The NeXus raw data file hierarchy is the consequence of some practical considerations. When looking at a beamline it is easy to 
discern different components: beam optic components, sample position, detectors and such. It is quite natural to mirror this physical 
separation with a logical arrangement of storing the data from each component in a separate group. This approach explains the 
list of beamline components in the NXinstrument group presented in Table\ref{rawfile}. 
As there can be multiple instances of the same kind of equipment, like slits or detectors, in a given beamline it becomes necessary
to add type information to the group name, which is provided by the NeXus class name. By convention NeXus class names start 
with the prefix NX. Each NeXus group describing a beamline component contains further groups and fields describing the component. 
A field can contain a single number, a text string or an array, as appropriate to the data to be described.  

The requirement to store multiple related experiments in the same file or to capture 
a complete workflow in a file causes the beamline component hierarchy to be pushed one level deeper into an NXentry 
group in the hierarchy. The NXentry  group thus represents one experiment (or a processed data entry, as will be discussed later). 
The NXentry group also holds the experiment metadata, such as the date and time at which it was performed. 

The user requirement to be able to access sample information and counting information such as monitors quickly 
pulled this information out of the beamline hierarchy and into the NXentry level.

Another requirement was to provide easy access to a default visualization of the data. This idea was realised with the 
NXdata type group at NXentry level. The NXdata group contains the data and axes data necessary to create a default plot of 
the data in the entry. This is data which will typically live in the NXdetector or other groups in the 
beamline component list. HDF-5 links are used to avoid duplicating data. A link 
is a pointer to data sitting at another place in the file hierarchy, like a symbolic link in a unix 
file system.  The NXdata group is intended to hold such links to the relevant pieces of information.

\subsubsection{Multiple Method Instruments}

Especially at X-ray sources, some beamlines implement multiple techniques in the same instrument. 
For example small-angle scattering (SAS) and powder diffraction in the same beamline are sometimes measured 
simultaneously (such as SAXS/WAXS). This poses a problem for an automatic data analysis tool to find its data 
for processing.  NeXus solves this problem through a scheme involving these steps:
\begin{enumerate}
\item All the data is stored in one NXentry hierarchy
\item At entry level, for each experimental method of the beamline an NXsubentry group is introduced. The 
  NXsubentry groups have the same hierarchy as the  NXentry group. NXsubentry holds only the data relevant 
  for the experimental method it describes. In order to avoid data duplication, links into the main NXentry 
  hierarchy are used. 
  % This is still awkward without referring to NXDL (which adds jargon).  NXDL is described later.
  NXsubentry usually only holds the 20-30 data items required by its corresponding NeXus 
  application definition.  
\end{enumerate} 
These steps are represented in the overview of Table \ref{multimethod}.

\begin{table}
\caption{Overview of the structure of a NeXus raw data file for an instrument with multiple methods \label{multimethod}}
\begin{tabular}{|p{1cm}|p{6.5cm}|}
\hline
\multicolumn{2}{|l|}{NXroot}\\
\hline
... &
\begin{tabular}{p{1.5cm}|p{4.8cm}}
\multicolumn{2}{l}{NXentry}\\
\hline
All data from multiple methods &
\begin{tabular}{p{1.5cm}|p{3.1cm}}
\multicolumn{2}{l}{NXinstrument}\\
\hline
...&
\begin{tabular}{p{3.0cm}}
NXsource\\
\hline
\ldots\\
\end{tabular}\\
\hline
\multicolumn{2}{l}{NXsample}\\
\hline
\multicolumn{2}{l}{NXuser}\\
\hline
\multicolumn{2}{l}{NXdata}\\
\hline
\multicolumn{2}{l}{\bf{NXsubentry}}\\
\multicolumn{2}{p{3.0cm}}{SAS specification}\\
\hline
\multicolumn{2}{l}{\bf{NXsubentry}}\\
\multicolumn{2}{p{3.0cm}}{powder diffraction specification}\\
\end{tabular}\\
\end{tabular}\\
\hline
\end{tabular}
\end{table}

\subsubsection{Scans}

Scans come in all shapes and sizes. Almost anything can be scanned against anything. 
An additional difficulty is that in practice, the number of scan points in the scan 
cannot be known in advance since it is possible that a scan may be interrupted or terminated
before its planned number of observations. 
Thus, it is a challenge to standardize a scan.
NeXus solves these difficulties through 
a couple of conventions and the use of a HDF-5 feature called unlimited dimensions. With the HDF-5 
unlimited dimensions feature, one axis of the data is allowed to expand without limit and 
the size of a data array does not need to be declared in advance. Data can be appended 
to an array along the unlimited dimension as required. 

Scans are stored in NeXus following these conventions: 
\begin{itemize}
\item Each variable varied or collected in the scan is stored at its appropriate place in the NeXus beamline 
 hierarchy as an array. The array's first dimension is the number of scan points. This is the unlimited dimension in 
 the implementation and data is appended at each scan point to the array. 
\item The NXdata group holds links to all the variables varied or collected during the scan. 
 This creates something equivalent or better than the tabular representation people are used to for scans. 
 The main detector data scan be plotted against any scanned parameter as well as against everything that was 
 deemed worth recording in addition to that, reading the NXdata group alone. 
\end{itemize}

NeXus allows multi dimensional scans too. This makes it very simple to produce meaningful slices through data 
volmes even with NeXus-agnostic software (like hdfview). Interrupting a multi-dimensional scan may, depending 
on the software used, leave some of the data in an uninitialised state (usually the HDF-5 fill value). 
This is behaviour is currently undefined in NeXus.

\subsubsection{Coordinate System and Positioning of Components}

For analysing data it is often necessary to know the exact position and orientation of beamline components. 
The first thing needed is a reference coordinate system. NeXus chose to use same coordinate system as the 
neutron beamline simulation software McStas\cite{mcstas}. 

For describing the placement and orientation of components, NeXus stores the same information as used for the 
same purpose in the Crystallographic Interchange Format (CIF)\cite{ITCVG}. CIF (and NeXus) stores the details 
of the translations and rotations necessary to move a given component from the zero point of the coordinate 
system to its actual position. As coordinate transformations are not commutative, the order of transformations 
must also be stored.

\begin{table}
\caption{Overview of the structure of a NeXus processed data file \label{procfile}}
\begin{tabular}{|p{1cm}|p{6.5cm}|}
\hline
\multicolumn{2}{|l|}{NXroot}\\
\hline
The root level of a NeXus file & 
\begin{tabular}{p{1.5cm}|p{4.8cm}}
\multicolumn{2}{l}{NXentry}\\
\hline
All data belonging to this processed data entry&
\begin{tabular}{p{1.5cm}|p{3.0cm}}
\multicolumn{2}{l}{NXprocess}\\
\hline
The data needed to describe this processing step&
\begin{tabular}{p{3.0cm}}
input:NXparameter\\
\hline
output:NXparameter\\
\hline
\end{tabular}\\
\hline
\multicolumn{2}{l}{NXsample}\\
\multicolumn{2}{p{3.0cm}}{All the information about the sample}\\
\hline
\multicolumn{2}{l}{NXdata}\\
\multicolumn{2}{p{3.0cm}}{The result data from the data processing including its axes}\\
\end{tabular}\\
\end{tabular}\\
\hline
\end{tabular}
\end{table}

\subsection{Processed Data}

On request of the user community, NeXus created a simplified structure for storing the result of data 
processing: be it reduction or analysis. In many cases even the reduced data is big enough to need 
an efficient binary representation. A good example is a tomography reconstruction. A tabular representation 
of the NeXus processed data file structure is given in Table \ref{procfile}. 

The hierarchy is much reduced as it is not important to carry all experimental information in the data 
reduction. In contrast to the raw data file structure, NXdata in the processed file structure is the place 
to store the results of the processing, together with its associated axes if the result is a multi-dimensional array.   
 
Information about the sample and instrument can be stored in NXsample and NXinstrument groups as required. 

In addition, there is a new structure to store details about the processing such as the program used, its version, 
the date of processing, and other metadata 
in the NXprocess group. The NXprocess group can hold additional NXparameter groups which are containers 
for storing the input and output parameters of the program used to perform the processing. 


\section{The NeXus Dictionary}

As can be seen from the discussion of the NeXus file hierarchy, NeXus arranges data into groups which have a 
type descriptor, the NeXus base class, associated with them. For each of these NeXus base classes, there 
exists a description of all the fields and groups possible within such a NeXus base class as well as definitions 
of the scientific meaning of each field. The collection of these NeXus base classes constitute the NeXus dictionary. 
The term class is a little misleading: NeXus base classes are not classes in strict object oriented notation, but 
are dictionaries of allowed names. A common misconception is that NeXus users have to give values for all the 
names in a NeXus base class. This is not the case; only those applicable to the instrument and situation at hand need 
to be written to the file.

The NeXus base classes are encoded in NeXus Description Language (NXDL).
NXDL is an XML file that specifies the content of the NeXus base class.

Procedures are in place to extend NeXus base classes quickly when necessary.


\section{NeXus Application Definitions}

The NeXus approach to standardization is unique for a standard. At first NeXus asks you to store 
as much relevant information about your data as possible. In a second step NeXus then defines a 
standard for a certain use case of NeXus, an application definition, wherein domain-specific fields 
for that use case are described as well as their locations within the NeXus file. This 
definition is the NeXus application definition. Use cases can be experimental techniques like SAS or 
powder diffraction, while processed data use cases such as tomography reconstructions or S(Q,Omega), 
or even administrative use cases such as archiving. This scheme works because data in a NeXus file 
can be searched for. Thus it is trivial to ignore data which is not of interest for a particular use case.

Another way to look at a NeXus application definition is as a contract between file writers and file consumers 
about the minimum content of the file necessary to fulfill that particular use case as well as a map of where the 
data should be located and its format.

NeXus application definitions are expressed in NXDL.  They may be parsed either by humans or by software and 
they may be validated for syntax and content.  The NXDL files are used to validate the structure of
NeXus data files. A tool exists to perform such validation.% can we please add some kind of reference to where the validation tool can be found?

A great number of candidate NeXus application definitions exist which were derived from our understanding 
of the technique described. For each of these, the NeXus team seeks community approval. 
Currently scientists 
from NeXus and the IUCr are nearly finished with a NeXus application definition for macromolecular crystallography.
CBFlib\cite{cbflib} is being extended to work with NeXus-MX format. This work will be published in another paper. 
Work on another NeXus application definition for reduced small-angle scattering data
is also in progress\cite{cansas}  by members of 
canSAS, NeXus, and the IUCr Commission on Small-Angle Scattering.


\section{Uptake of NeXus} 

NeXus is already in use as the main data format at facilities like Soleil, Diamond, SINQ, SNS, Lujan/LANL 
and KEK. Other facilities like ISIS, DESY and the muSR community are in the process of moving towards 
NeXus as  data format. The adoption of NeXus took time. The reason is that NeXus is often chosen whenever 
a facility starts operation or undergoes major refurbishments. For those facilities where there is an existing and working 
pipeline from data acquisition to data analysis,  the resources are usually lacking to move 
towards NeXus.

\section{NeXus Governance}

The development of NeXus is overseen by the NeXus International Advisory Committee (NIAC). 
In the NIAC, most relevant facilities are represented. It is easy to join the NIAC if 
there is justified interest. 

\section{Backwards Compatibility}

In the past, NeXus supported data files in HDF-4, HDF-5, and XML file formats. 
To support writing software to write and read these file formats, the NeXus
application/programmer interface (NeXus-API) was provided.
This API still exists and is maintained at a bug fix level.
On request of the community, we now concentrate our efforts only
on the HDF-5 files and tools.

\section{Summary}

NeXus has matured considerably over the last 10 years and is now in use in many facilities. NeXus 
is flexible enough to accommodate a wide variety of instruments and scientific applications. 
Yet it is efficient enough to 
handle the data coming from modern high speed detectors. For more information,
do not hesitate to consult the NeXus WWW--site\cite{nxwww} or to contact
the members of the NIAC. 



\nocite{*}
\bibliography{nexus14aip}% Produces the bibliography via BibTeX.

\end{document}
%
% ****** End of file aipsamp.tex ******
